\documentclass{beamer}
\usepackage{listings}
\lstset{
%language=C,
frame=single, 
breaklines=true,
columns=fullflexible
}
\usepackage{subcaption}
\usepackage{url}
\usepackage{commath}
\usepackage{tikz}
\usepackage{pgfplots}
\pgfplotsset{compat=1.17}
\usepackage{tkz-fct}
\usepackage{mathrsfs}
\usepackage{txfonts}
\usepackage{tkz-euclide} 
\usetikzlibrary{calc,math}
\usepackage{float}
\providecommand{\brak}[1]{\ensuremath{\left(#1\right)}}
\renewcommand{\vec}[1]{\mathbf{#1}}
\providecommand{\pr}[1]{\ensuremath{\Pr\left(#1\right)}}
\usepackage[export]{adjustbox}
\usepackage[utf8]{inputenc}
\usepackage{amsmath}
\usetheme{Boadilla}
\title{Numerically Efficient Fourier-Based Technique for Calculating Error Probabilities with Intersymbol Reference}
\subtitle{}
\author{Vamsi Preetham(CS20BTECH11058)}
\institute{Indian Institute of Technology Hyderbad}
\date{26-6-2021}

\begin{document}

\begin{frame}
\titlepage
\end{frame}

\begin{frame}
\frametitle{Introduction}
Here we propose a numerically efficient technique for calculating of the probability the symbol error for arbitary coherent modulation schemes in the presence of Intersymbol Reference and additive noise.
\end{frame}

\begin{frame}
\frametitle{Abbreviations and Notations}
\begin{description}
\item[ISI - Intersymbol Interference]
\item[CAD - Computer-Aided Design]
\item[M-PSK - M-ary Phase-Shift Keying]
\item[M-QAM - M-ary Quadrature Amplitude Modulation]
\item[QPSK - Quadrature Phase Shift Keying]
\item[BPSK - Binary Phase Shift Keying]
\end{description}

\end{frame}


\begin{frame}
\frametitle{Outline}
\begin{itemize}
\item The effects of Intersymbol Interference(ISI) can degrade the probability of error performance of the receivers.
\item For determining the performance of the receiver $P_e$ is evaluated over a particular sequence of ISI symbols and then averaged.
\item But this technique is computationally expensive so cosiderable work is done for more efficient techniques.
\item Here we present a method that can be applied to 1 and 2-D coherent signaling formats.
\item The probability of correctly decoding a particular symbol is the inverse Fourier transform of the windowed characteristic function.
\item Sampling theorem , truncation and Appropriate spectral windowing are used for numerical evaluation of the integral.
\end{itemize}
\end{frame}
\begin{frame}
\frametitle{Problem Statement and General Solution}
\begin{block}{Receiver Output}
 Sample solution of a coherent receiver $y$ in the presence of ISI and noise
    \begin{align}
     y =  \sum_{{k=-N}_{k\not\equiv0}}^{N} b_ka_k + b_0a_0 +n 
    \end{align} 
    where
\begin{itemize}
    \item $b_k$'s represent the transmitted data sequence in which each symbol is one of the M-possible equally probable symbols.
    \item $a_k$'s are ISI coefficients.
    \item $b_0$ current symbol to be detected.
\end{itemize}
\end{block}
\end{frame}

\begin{frame}{}
    \begin{block}{Probability of correct decision}
     The probability of making a correct decision when $j^{th}$ element $s_j$ is sent is 
     where
     \begin{align}
         P_{cj} = \pr{y\in D_j | b_0 = s_j}
     \end{align}
     \begin{itemize}
         \item $D_j$ is the decision region in the complex plane corresponding to $s_j$. 
     \end{itemize}
     The probability of the error is 
     \begin{align}
         P_E = 1 - \frac{1}{M}\sum_{j=1}^{M} P-{c_{j}}.
     \end{align}
   Here $y$ is written in terms of in-phase and quadrature components as $y=y_I+iy_Q$ , $h_Y(\omega_I,\omega_Q)$ is it's bivariate characteristic notation,
    \begin{align}
        h_Y(\omega_I,\omega_Q)=h_{ISI}(\omega_I,\omega_Q)h_N(\omega_I,\omega_Q)e^{i(\omega_IRe{s_ja_0}+\omega_Q{Ims_ja_0})}
    \end{align}
    \end{block}
\end{frame}

\begin{frame}{}
\begin{block}{Probability Density Function}
 The Probability density function of $y$ is given by
 \begin{align}
     f_Y(y_I,y_Q) = \frac{1}{(2\pi)^2}\int_{-\infty}^{\infty}\int_{-\infty}^{\infty} h_Y(\omega_I,\omega_Q)e^{-i(\omega_iy _I+\omega_Qy_Q)}\,d\omega_I \,d\omega_Q
     \end{align}
As,
    \begin{align}
        P_{cj}=\int_{}^{}\int{D_j}^{}f_Y(y_I,y_Q)\,dy_i \,dy_Q
    \end{align}
We combine $(5)$ and $(6)$ to
\begin{align}
P_{c_j}=\frac{1}{2\pi}\int_{-\infty}^{\infty}\int_{-\infty}^{\infty}K_j(\omega_I,\omega_Q)h_{ISI}(\omega_I,\omega_Q)h_N(\omega_I,\omega_Q)e^{i(\omega_IRe\{s_ja_0\}+\omega_QIm\{s_ja_o\})}\\ d\omega_Id\omega_Q    
\end{align}
\end{block}
\end{frame}
\begin{frame}{}
    \begin{block}{Fourier Equation}
     Here $K_j(\omega_I,\omega_Q)$ is called the Fourier form and the equation is called the Inverse Fourier Transformation.
     \begin{itemize}
         \item The probability and the spectral domains are equivalently stated with the relations
     \end{itemize}
     \begin{align}
        P_j(t_I,t_Q)=F^{-1}\{k_j(\omega_I,\omega_Q)h_{ISI}(\omega_I,\omega_Q)h_N(\omega_I,\omega_Q)\} 
     \end{align}
     Or equivalently
     \begin{align}
        P_j(t_I,t_Q)=D_j(t_I,t_Q)f_ISI(t_I,t_Q)f_N(t_I,t_Q)
     \end{align}
     \begin{itemize}
         \item  Where $f_{ISI}$ and $f_{N}$ are the probability density functions of the random variables representing ISI and Noise.
         \item Sampling theorem and Equations(8) and (9) are used to compute the Fourier Integrals.
     \end{itemize}
    \end{block}
\end{frame}
\begin{frame}{}
    \begin{block}{Poisson's Sum}
      With rectangular grid of Spectral Resolution $\delta w_I=\delta w_Q = \frac{2\pi}{T}$, the Poisson's sum is
      \begin{align}
          \sum_{m_Q=-\infty}^{\infty}\sum_{m_Q=-\infty}^{\infty}P_j(t_I-m_IT,t_Q-m_QT)=\frac{1}{T^2}\sum_{t_Q=-\infty}^{\infty}\sum_{t_Q=-\infty}^{\infty}K_j(\frac{2\pi}{T}l_I,\frac{2\pi}{T}l_Q)\\.h_{ISI}(\frac{2\pi}{T}l_I,\frac{2\pi}{T}l_Q).h_N(\frac{2\pi}{T}l_I,\frac{2\pi}{T}l_Q).e^{i(l_It_I+l_Qt_Q)\frac{2\pi}{T}}
      \end{align}
      But $eq-(11)$ is not defined in some regions because the decision region $D_f$ can be of infinite duration in the $(t_I,t_Q)$n plane in infinite $P_j$. To correct this decision region is modified with a ideal brick-wall window.
    \end{block}
\end{frame}
\begin{frame}{}
    \begin{block}{Corrected Probability}
    After truncation the calculated probability of correctly detecting the $j^{th}$ symbol is  \begin{align}
        \vec{P_{cj}}=\frac{1}{T^2}\sum_{t_Q=-P}^{P}\sum_{t_Q=-P}^{P}\vec{K_j}\prod_{k=-N}^{k=N}h_{ISIk}(\omegal_I,\omega_Q).h_N(\omegal_I,\omegal_Q)e^{i(\omegal_IRe\{s_ja_0\}+\omegal_QIm\{s_ja_o\})}\\d\omega_Id\omega_Q 
    \end{align}
    ANd $P_c_j=\vec{P_c_j}$+error(error=$\epsilon_1-\epslion_2+\epsilon_3)$\\
    The first component $\epsilon_1$ is the result of the modified design region and can be written as 
    \begin{align}
        \epsilon_1=(D_j(t_I,t_Q)-\vec(D_J)(t_I,t_Q))*f_{ISI}(t_I,t_Q)*f_N(t_I,t_Q)
    \end{align}
evaluated at $t_I=a_{0I}-a_{0Q}$ and $t_Q=a_{0I}+a_{0Q}$ 
\end{block}
\end{frame}

\begin{frame}{}
    \begin{block}{errors}
    The second component $\epsilon_2$ is due to aliasing and is determined by applying $\vec{D_j}$ to eq(9) and using eq(10) as,
     \begin{align}
     \epsilon_2=\sum{}{} \sum{}{} \vec{P_j}(t_I-m_IT,t_Q-m_QT)
     \end{align}
     Finally $\epsilon_3$ is due to spectral truncation 
     \begin{align}
         |{\epsilon_3}|<\frac{1}{T^2}|{\vec{k_j}(\omega l_I,\omega l_Q)}||{h_{ISI}(\omega l_I,\omega l_Q)}||{h_N(\omega l_I,\omega l_Q)}|
     \end{align}
     Note that error should be much lesser than $P_{cj}$
    \end{block}
\end{frame}
\begin{frame}{Error Probability for QPSK}
\begin{block}{QPSK}
    The characteristic function of the $k^{th}$ interfernce component where $b_k$'s represent intake and quadratic components $(b_k=\pm1\pm i)$ is given by
    \begin{align}
        h_{ISIk}(\omega_I,\omega_Q)=cos(\omega_Ia_{kI}+\omega_Qa_{kQ})cos(\omega_Ia_{kQ}-\omega_Qa_{kI})
    \end{align}
    For noise both inphase and quadratic components are independent and identically distributed
    \begin{align}
        h_N(\omega_I,\omega_Q)=e^{-\frac{(\omega_I^2+\omega_Q^2)\sigma^2}{2}}
    \end{align}
    The fourier transformation is
    \begin{align}
        \vec(K_{1})(\omega_I,\omega_Q)=\frac{T^2}{4}sinc\frac{\omega_IT}{4\pi}sinc{\omega_QT}{4\pi}e^{-i(\omega_I+\omega_Q)\frac{t}{4}}    
    \end{align}
    where sinc(x)=\frac{sin\pi*x}{\pi*x}    
\end{block}
\end{frame}

\begin{frame}{}
    \begin{block}{parameters}
        $\epsilon$ is a user-specified parameter where $\gamma$ and $\phi$ and P the spectral truncation function are 
        \begin{align}
            \epsilon=\phi(\frac{\frac{-T}{2}+\gamma}{\sigma})\\
            \gamma=\sum_{K=-N}^{N}(|a_{kI}|+a_{kQ}|)\\
            \phi=(\sqrt{2\pi})^{\frac{-1}{2}}\sum_{-\infty}^{x}e^{\frac{-y^2}{2}dy}\\
            P=\ceil(\frac{LT}{2\pi\sigme})
        \end{align}
   The total numerical error is 
        \begin{align}
            |error|<30\epsilon
        \end{align}
    \end{block}
\end{frame}

\begin{frame}
    \begin{block}
    The probability of error is $P_E=1-P_{C1}$. The probability of error for QPSK is
    \begin{align}
        \vec{P_E}=1-\frac{1}{4}\sum_{l_Q=_P}^{P}\sum_{l_Q=_P}^{P}sinc\frac{l_I}{2}sinc\frac{l_Q}{2}\prod_{k=-N}{N}h_{ISIk}(\omega l_I,\omega_Q).h_N(\omega l_I,\omega l_Q)\\.cos(\omega l_I(a_{0I}-a_{0Q}-\frac{T}{4})+\omega l_Q(a_{0I}+a_{0Q}-\frac{T}{4}
    \end{align}. The probability of bit error for BPSK is $P_B=1-\sqrt{1-P_E}$.
    Table 1 contains the calculated probability probability for symbol error for r=0(induces crosstalk between in-phase and quadrature channels) and for various values of N and table 2 is for r=0.25.\\\\
    This is a simple numerically efficient techniques for calculating probability of error with ISI and additive noise.
    \end{block}
\end{frame}
\begin{figure}[htp]
    \centering
    \includegraphics[width=\columnwidth]{WhatsApp Image 2021-07-01 at 16.41.22.jpeg}
    \caption{Table 1 and 2 $P_E$ vs r curve}
    \label{fig:galaxy}
\end{figure}

\end{document}

